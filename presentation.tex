\documentclass{beamer}

\usepackage[utf8]{inputenc}
\usepackage{default}
\usepackage{hyperref}

\usetheme{Torino}


\begin{document}

\title{Embedding Information Literacy Principles in Interface Design}
\author{Jane Sandberg}
\institute{Linn-Benton Community College}


\begin{frame}
 \titlepage
\end{frame}

\begin{frame}{Types of searches}
 \begin{block}{Stats from our discovery layer} 
 \begin{itemize}
  \item 13\% known-item searches (n=620)
  \item 87\% exploratory searches (n=4189)
 \end{itemize}
 \end{block}

  \begin{block}<2->{What happens after you execute your search?} 
 \begin{itemize}
  \item After the patron gets the results of their known item search, they are almost done
  \item After the patron gets the results of an exploratory search, their job is just beginning
 \end{itemize}
 \end{block}
\end{frame}

\begin{frame}{Typical, unhelpful ways to re-formulate queries}
 \begin{itemize}
  \item Moving words around to be in a different order
  \item Changing short words for each other (e.g. ``a'' becomes ``the'')
  \item Switching singular $\rightarrow$ plural, changing verb tenses
  \item Changing a single word for its synonym
 \end{itemize}

 Patrons are hesitant to use broader/narrower terms, related terms, radically change their search
 
\end{frame}


\begin{frame}{Searching as Strategic Exploration}
From the ACRL Framework for Information Literacy in Higher Education:

\vfill

``Searching for information is often nonlinear and iterative, requiring the evaluation of a range of information sources and the mental flexibility to pursue alternate avenues as new understanding develops.''
 
\end{frame}

\begin{frame}{Searching as Strategic Exploration: Dispositions}
Learners who are developing their information literate abilities

\begin{itemize}
 \item exhibit mental flexibility and creativity
 \item understand that first attempts at searching do not always produce adequate results
 \item seek guidance from experts, such as librarians, researchers, and professionals
 \item recognize the value of browsing and other serendipitous methods of information gathering
 \item persist in the face of search challenges, and know when they have enough information to complete the information task
\end{itemize}

\vfill
Note: these dispositions are part of a longer list

\end{frame}

\begin{frame}
 We can (and do) foster these dispositions when we teach and do reference work.
 
 \vspace{1in}
 
 But is there a way to foster them when the patron is researching on their own?  Can our search interfaces encourage patrons to develop these dispositions?
\end{frame}


\begin{frame}{Some preliminary thoughts}

\begin{itemize}
 \item Make it as easy as possible to submit the initial search
 \item Offer example searches for inspiration, but taking care not to constrain patrons
 \item Be intentional about the feedback we present along with the search results
 \item Facets facets facets
 \item Make it easy to seek guidance
 \item Make the patrons slooooow dooooown
\end{itemize}


\end{frame}

\begin{frame}{If you have a piece of this puzzle}
I would love to talk to you / email
\vfill
sandbej [at] linnbenton [dot] edu
\end{frame}



\end{document}
